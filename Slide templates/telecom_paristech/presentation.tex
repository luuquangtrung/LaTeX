\documentclass{beamer}


% *** TPT STYLE ***

\definecolor{colorframetitle}{RGB}{191,18,56} % Title of frames
\definecolor{redbox}{RGB}{191,18,56} % 
\definecolor{blackbox}{RGB}{0,0,0} % 
\definecolor{brownbox}{RGB}{128,99,90} % 

\usepackage[absolute,overlay]{textpos}
\usepackage{listings}
\usepackage{hyperref}

\setlength{\TPHorizModule}{1mm}
\setlength{\TPVertModule}{1mm}

\usepackage{tikz}
\usetikzlibrary{decorations.pathreplacing,calc}
\usetikzlibrary{arrows,shapes,snakes,automata,backgrounds,petri}
\usepackage{scalefnt}
\usepackage[siunitx]{circuitikz} % Loading circuitikz with siunitx option
\usepackage{pgfplots}


\newcommand{\tikzmark}[1]{\tikz[overlay,remember picture] \node (#1) {};}

\tikzstyle{mybox} = [draw=redbox, fill=redbox!20, very thick,
rectangle, rounded corners, inner sep=10pt, inner ysep=20pt]
\tikzstyle{fancytitle} =[fill=redbox, text=white, rectangle]


\newcommand{\MyLogo}{
  %\begin{textblock}{14}(117.2,0.7)
  \begin{textblock}{14}(117.8,86)
    \includegraphics[width=1cm]{figures/tpt}
  \end{textblock}
}

\usepackage{beamerthemesplit}

\setbeamercolor{itemize item}{fg=redbox}
\setbeamercolor{structure}{fg=redbox, bg=red}
\setbeamercolor{block title}{bg=brownbox,fg=white}
\setbeamercolor{block title alerted}{bg=redbox,fg=white}
\setbeamercolor{block body alerted}{bg=brownbox!0,fg=black}
\setbeamercolor{block title example}{bg=black, fg=white}
\setbeamercolor{palette primary}{fg=black,bg=white} % changed this
\setbeamercolor{palette secondary}{use=structure,fg=structure.fg!100!white} % changed this
\setbeamercolor{palette tertiary}{use=structure,fg=structure.fg!100!white} % changed this
\setbeamercolor*{palette quaternary}{fg=black,bg=white} % outline on top left
\setbeamercolor{background canvas}{bg=white, fg=black} 
\setbeamercolor{frametitle}{fg=colorframetitle}


% First  frame
\newcommand{\RectanglesOfMainSlide}{%
  \raisebox{0mm}[0pt][0pt]{%
    \begin{pgfpicture}{0mm}{0mm}{0mm}{0mm}
      \pgfsetlinewidth{5mm}
      \color{redbox}
      \pgfline{\pgfpoint{-4mm}{-12mm}}{\pgfpoint{24mm}{-12mm}}
      \color{blackbox}
      \pgfline{\pgfpoint{24mm}{-12mm}}{\pgfpoint{52mm}{-12mm}}
      \color{brownbox}
      \pgfline{\pgfpoint{52mm}{-12mm}}{\pgfpoint{80mm}{-12mm}}
\end{pgfpicture}}}

\newcommand{\makeFirstFrame}{
  \setbeamertemplate{footline}{} 
  \frame[plain]{
    \begin{columns}[c]
      \column{3cm}
      \vspace{-2cm}\\
      \includegraphics[width=2.5cm]{figures/tpt}
      \\Institut\\ Mines-Telecom
      \column{7cm}
      \vspace{1cm}\\
      \LARGE{\textbf{\theTitle}}\\
      \vspace{0.5cm}
      \normalsize{\theAuthors}\\
      \vspace{0.5cm}
      \normalsize{\theConferenceAndPlace}\\
      \vspace{-1cm}
      \RectanglesOfMainSlide
    \end{columns}
  }
  \activateFootline
}


% Frames decoration

\newcommand{\RectanglesBeforeTitle}{%
  \raisebox{0mm}[0pt][0pt]{%
    \begin{pgfpicture}{0mm}{0mm}{0mm}{0mm}
      \pgfsetlinewidth{5mm}
      \color{redbox}
      \pgfline{\pgfpoint{-2mm}{2.2mm}}{\pgfpoint{4mm}{2.2mm}}
      \color{blackbox}
      \pgfline{\pgfpoint{4mm}{2.2mm}}{\pgfpoint{10mm}{2.2mm}}
      \color{brownbox}
      \pgfline{\pgfpoint{10mm}{2.2mm}}{\pgfpoint{16mm}{2.2mm}}
\end{pgfpicture}}}

\setbeamertemplate{frametitle}{
  \begin{columns}[t]
    \column{16mm}
    \RectanglesBeforeTitle 
    \column{10.7cm}
    \strut\textbf{\insertframetitle}\strut
  \end{columns}
}

% Foot line

\newcommand{\Ffootline}{
  \MyLogo
  \begin{tikzpicture}
    \fill [color=white, fill=redbox] (-1, -0.05) rectangle (1, 0.30);
    \node[white, right] (note1) at (-1, 0.10) {\insertframenumber/\inserttotalframenumber};
    \node[white, left] (note1bis) at (0.98, 0.10) {\theDate};
    \fill [color=white, fill=blackbox] (1.05, -0.05) rectangle (4.5, 0.30);
    \node[white, align=center] (note2) at (2.77, 0.12) {Institut Mines-Telecom};
    \fill [color=red, fill=brownbox] (4.55, -0.05) rectangle (10.75, 0.30);
    \node[white, align = center] (note3) at (7.65, 0.10) {\theTitle};

    %\node[white] (note3) at (7.5, 0.10) {\theTitle};
  \end{tikzpicture}
}

\newcommand{\activateFootline}{
  \setbeamertemplate{footline}{
    \usebeamerfont{structure}
    \Ffootline
  }
}


% *** END OF TPT STYLE ***


%remove navigation symbols
\setbeamertemplate{navigation symbols}{}

% To show the outline at the beginning of each section
\AtBeginSection[]{
  \begin{frame}
    \frametitle{Outline}
    %\begin{center}{\LARGE Outline }\end{center}
    \tableofcontents[currentsection,hideothersubsections]
  \end{frame} 
}

%\newcommand{\LinkToMethodo}{
%\begin{textblock}{25}(102,89)
%  \hyperlink{METHODO}{\beamergotobutton{Back to methodology}}
% \end{textblock}
%}

%\lstset{language=C,basicstyle=\scriptsize,keywordstyle=\color{red}\bfseries,  commentstyle=\color{blue}\textit,stringstyle=\color{green}\ttfamily,labelstyle=\tiny, showspaces=false,showstringspaces=false}

\newcommand{\mytilde}{\raise.17ex\hbox{$\scriptstyle\mathtt{\sim}$}}

\newcommand{\tikzgrid}{
  \begin{pgfonlayer}{background}
    \draw[gray!50]
    (current bounding box.south west)
    grid[step=.2] (current bounding box.north east);
    \draw[red!50]
    (current bounding box.south west)
    grid (current bounding box.north east);
  \end{pgfonlayer}
}

\newcommand*{\ExtractCoordinate}[3]{\path (#1); \pgfgetlastxy{#2}{#3};}%

\newdimen\tlx
\newdimen\tlx
\newdimen\brx
\newdimen\bry


%% To FILL to customize presentation with the TPT style

\newcommand{\theTitle}{Presentation}
\newcommand{\theAuthors}{Quang-Trung LUU {quang-trung.luu@u-psud.fr}}
\newcommand{\theConferenceAndPlace}{M2R Multimedia Networking}
\newcommand{\theDate}{Sept. 2016}


\begin{document}


\makeFirstFrame

\frame{
  \frametitle{Outline}
  \tableofcontents
}

\setbeamertemplate{blocks}[rounded][shadow=true]

\section{Introduction}

\frame[containsverbatim]{
  \frametitle{Making A TPT Presentation with Beamer!}


  \begin{alertblock}{It looks just like the powerpoint model, right?}
    Well, I've made my best\ldots\\
    It can obvisouly be improved, please, send me your suggestions
  \end{alertblock}

  \begin{exampleblock}{How to do?}
    Simply insert this style at the top of your beamer presentations
  \end{exampleblock}

  \begin{block}{And more precisely?}
    Fill the title, authors conference and date information at the end of the style description
  \end{block}
}

\section{Already the conclusion}

\frame[containsverbatim]{
  \frametitle{Questions?}

  Contact me!

  \begin{itemize}
    \item http://perso.telecom-paristech.fr/\mytilde apvrille/
    \item http://ttool.telecom-paristech.fr/
  \end{itemize}
}

\section{Try to create a circuit}



\begin{circuitikz}
	\draw
	% Drawing a npn transistor
	(0,0) node[npn](npn1){} 
	% Making connections from transistor using relative coordinates
	(npn1.E) node[right=7mm, above=5mm]{2N2222} % Labelling the transistor
	(npn1.B) -- ++(-1,0) to [R,l_=10<\kilo\ohm>,*-*] ++(0,-3)  
	(npn1.B) -- ++(-3,0) to [C,l_=100<\nano\farad>] ++(0,-3) node(gnd1){}
	(npn1.E) to [R,l_=10<\kilo\ohm>,*-*] (0,-3)
	(npn1.E) -- ++(2,0) to [C,l=10<\pico\farad>,*-*] (2,-3)
	(npn1.B) -- ++(-1,0) to [R,l_=10<\kilo\ohm>,*-] ++(0,3) node(con1){}
	(npn1.C) to [L,l_=150<\micro\henry>,*-] (0,3) 
	(npn1.C) -- ++(2,0) to [C,l=10<\pico\farad>,*-*] ++(0,-1.5)
	% Drawing shorts and ground connection
	(-1,3)to[short,*-o] (-1,4) node[right]{$V_{DD}=6 VDC$} % Power supply
	% Output sinusoidal waveform at approximately 50 MHz
	(npn1.C) -- ++(4,0) to [short,-o]
	  ++(0,0) node[right]{$V_o (\approx \SI{50}{\MHz})$}
	(0,-3) node[ground]{}% Define this node as ground
	(gnd1) ++(0,0) to[short,-o] ++(7.85,0)
        (con1)to[short] ++(1.85,0)
	;
\end{circuitikz}


\section{Draw a graph}

\begin{tikzpicture}
  \begin{axis}[title  = Contributions per category
                          at LaTeX-Community.org,
    xbar,
    y axis line style = { opacity = 0 },
    axis x line       = none,
    tickwidth         = 0pt,
    enlarge y limits  = 0.2,
    enlarge x limits  = 0.02,
    symbolic y coords = {LaTeX, Tools, Distributions, Editors},
    nodes near coords,
  ]
  \addplot coordinates { (57727,LaTeX)         (5672,Tools)
                         (2193,Distributions)  (11106,Editors) };
  \addplot coordinates { (14320,LaTeX)         (1615,Tools)
                         (560,Distributions)   (3075,Editors)  };
  \legend{Topics, Posts}
  \end{axis}
\end{tikzpicture}

\end{document}
